\documentclass{article}
\usepackage[utf8]{inputenc}
\usepackage{geometry}
\usepackage{graphicx}
\usepackage{hyperref}
\usepackage{booktabs}
\usepackage{array}
\usepackage{longtable}
\usepackage{colortbl}
\usepackage{xcolor}
\usepackage{enumitem}
\usepackage{listings}

\geometry{margin=1in}

\title{delim[[.vars.title]]}
\author{delim[[.vars.author]]}
\date{delim[[.vars.date]]}

\begin{document}

\maketitle

\section{Document Information}
delim[[.vars.description]]

\section{Company Overview}
\begin{itemize}
\item Total Employees: delim[[.vars.company.total_employees]]
\item Total Departments: delim[[.vars.company.total_departments]]
\item Total Projects: delim[[.vars.company.total_projects]]
\item Total Skills: delim[[.vars.company.total_skills]]
\item Average Salary: \$${delim[[.vars.company.avg_salary]]:,}
\item Active Employees: delim[[.vars.company.active_employees]]
\item Inactive Employees: delim[[.vars.company.inactive_employees]]
\end{itemize}

\section{Employee Directory}
\begin{longtable}{|p{3cm}|p{3cm}|p{2.5cm}|p{2cm}|p{2cm}|p{2cm}|}
\hline
\textbf{Name} & \textbf{Position} & \textbf{Department} & \textbf{Salary} & \textbf{Experience} & \textbf{Status} \\
\hline
\endhead

delim[[range .complex.employees]]
delim[[.name]] & delim[[.position]] & delim[[.department]] & \$${delim[[.salary]]:,} & delim[[.experience]] years & delim[[if .active]]Active\delim[[else]]Inactive\delim[[end]] \\
\hline
delim[[end]]

\end{longtable}

\section{Department Statistics}
\begin{table}[h]
\centering
\begin{tabular}{|l|c|c|}
\hline
\textbf{Department} & \textbf{Head Count} & \textbf{Average Salary} \\
\hline
delim[[range .complex.departments]]
delim[[.name]] & delim[[.head_count]] & \$${delim[[.avg_salary]]:,} \\
\hline
delim[[end]]
\end{tabular}
\caption{Department Overview}
\end{table}

\section{Project Portfolio}
\begin{longtable}{|p{3cm}|p{2cm}|p{2cm}|p{2cm}|p{2cm}|p{2.5cm}|p{2.5cm}|}
\hline
\textbf{Project} & \textbf{Status} & \textbf{Priority} & \textbf{Team Size} & \textbf{Budget} & \textbf{Start Date} & \textbf{End Date} \\
\hline
\endhead

delim[[range .complex.projects]]
delim[[.name]] & delim[[.status]] & delim[[.priority]] & delim[[.team_size]] & \$${delim[[.budget]]:,} & delim[[.start_date]] & delim[[.end_date]] \\
\hline
delim[[end]]

\end{longtable}

\section{Skills Matrix}
\begin{table}[h]
\centering
\begin{tabular}{|p{4cm}|p{10cm}|}
\hline
\textbf{Employee} & \textbf{Skills} \\
\hline
delim[[range .complex.employees]]
delim[[.name]] & delim[[range .skills]]delim[[.]]\delim[[end]] \\
\hline
delim[[end]]
\end{tabular}
\caption{Employee Skills Matrix}
\end{table}

\section{Department Skills}
\begin{table}[h]
\centering
\begin{tabular}{|l|p{8cm}|}
\hline
\textbf{Department} & \textbf{Skills} \\
\hline
delim[[range .complex.departments]]
delim[[.name]] & delim[[range .skills]]delim[[.]]\delim[[end]] \\
\hline
delim[[end]]
\end{tabular}
\caption{Department Skills Overview}
\end{table}

\section{Project Technologies}
\begin{table}[h]
\centering
\begin{tabular}{|l|p{8cm}|}
\hline
\textbf{Project} & \textbf{Technologies} \\
\hline
delim[[range .complex.projects]]
delim[[.name]] & delim[[range .technologies]]delim[[.]]\delim[[end]] \\
\hline
delim[[end]]
\end{tabular}
\caption{Project Technology Stack}
\end{table}

\section{All Skills Inventory}
\begin{table}[h]
\centering
\begin{tabular}{|p{3cm}|p{3cm}|p{3cm}|p{3cm}|p{3cm}|p{3cm}|}
\hline
delim[[range .complex.all_skills]]
delim[[.]] & delim[[end]]
\end{tabular}
\caption{Complete Skills Inventory}
\end{table}

\section{Salary Analysis}
\begin{table}[h]
\centering
\begin{tabular}{|l|c|c|c|}
\hline
\textbf{Department} & \textbf{Min Salary} & \textbf{Max Salary} & \textbf{Average} \\
\hline
delim[[range .complex.departments]]
delim[[.name]] & \$${delim[[.min_salary]]:,} & \$${delim[[.max_salary]]:,} & \$${delim[[.avg_salary]]:,} \\
\hline
delim[[end]]
\end{tabular}
\caption{Salary Analysis by Department}
\end{table}

\section{Project Status Summary}
\begin{table}[h]
\centering
\begin{tabular}{|l|c|c|}
\hline
\textbf{Status} & \textbf{Count} & \textbf{Percentage} \\
\hline
delim[[range .complex.project_status]]
delim[[.status]] & delim[[.count]] & delim[[.percentage]]\% \\
\hline
delim[[end]]
\end{tabular}
\caption{Project Status Distribution}
\end{table}

\section{Employee Experience Levels}
\begin{table}[h]
\centering
\begin{tabular}{|c|c|}
\hline
\textbf{Experience Range} & \textbf{Employee Count} \\
\hline
delim[[range .complex.experience_levels]]
delim[[.range]] & delim[[.count]] \\
\hline
delim[[end]]
\end{tabular}
\caption{Experience Distribution}
\end{table}

\section{Table Generation Examples}

This document demonstrates various ways to fill LaTeX tables with data:

\begin{enumerate}
\item \textbf{Simple Tables}: Basic tabular data with headers
\item \textbf{Long Tables}: Multi-page tables with automatic page breaks
\item \textbf{Complex Data}: Nested objects and arrays
\item \textbf{Conditional Formatting}: Status indicators and conditional content
\item \textbf{Number Formatting}: Currency and number formatting
\item \textbf{Range Loops}: Dynamic content generation from arrays
\item \textbf{Nested Loops}: Complex data structures with multiple levels
\end{enumerate}

\section{Template Syntax Examples}

\subsection{Basic Table with Range Loop}
\begin{verbatim}
\begin{tabular}{|l|c|}
\hline
\textbf{Name} & \textbf{Value} \\
\hline
delim[[range .complex.employees]]
delim[[.name]] & delim[[.salary]] \\
\hline
delim[[end]]
\end{tabular}
\end{verbatim}

\subsection{Conditional Content}
\begin{verbatim}
delim[[if .active]]Active\delim[[else]]Inactive\delim[[end]]
\end{verbatim}

\subsection{Nested Loops}
\begin{verbatim}
delim[[range .complex.employees]]
delim[[.name]]: delim[[range .skills]]
delim[[.]]\delim[[end]]
delim[[end]]
\end{verbatim}

\section{Best Practices}

\begin{itemize}
\item Use \texttt{longtable} for tables that might span multiple pages
\item Use \texttt{booktabs} package for professional table formatting
\item Format numbers and currency appropriately
\item Use conditional statements for status indicators
\item Organize data hierarchically for better template readability
\item Test with different data sizes to ensure table formatting
\end{itemize}

\end{document}